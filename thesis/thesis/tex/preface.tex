\chapter*{Introduction}
\addcontentsline{toc}{chapter}{Introduction}

\section{Globalization and Containers}
The nature of today’s world is greatly influenced by globalization,  
which not only connects economies and societies across continents, 
but also has a profound impact on cultures \cite{zizek1997}.\\

One of the main drivers of globalization is the transportation of goods. 
In order to transport goods effectively, containers are essential, 
as they allow for the easy movement of goods using various methods of transportation. 
They are used in trucks, ships, trains, and even airplanes, 
facilitating a smooth transition of goods when changing the mode of transport. 
The use of containers has tremendously reduced handling times and costs, making the global supply chain much more efficient \cite{bernhofen2016}.\\

Since the cost of shipping is typically calculated based on the number of containers, their volume, and their weight, 
any unused space consequently results in increased expenses. 
In large-scale logistics, where massive quantities of containers are shipped daily, 
even small improvements in packing efficiency can lead to significant cost savings as well as reduction in emissions. 
For example, in 2018, containers imported into the United States were, on average, loaded to only about 65\% of their capacity \cite{obyrne2021}.\\

\section{Container Packing Problems in Computer Science}
In the context of computer science and optimization, container packing problems generally revolve around three key types \cite{martello2000}:\\

\subsection{3D Bin Packing Problem}
The goal is to pack a set of rectangular boxes into the smallest possible number of containers, considering all three dimensions of the objects and preventing them from colliding.\\

\subsection{Knapsack Loading Problem}
Each box has an associated value, and the objective is to select a subset of boxes that fits into a single container while maximizing the sum of their values.\\

\subsection{Container Loading Problem}
The aim is to pack all boxes into a single container, which has infinite length, while minimizing the total volume used.\\

All of these problems are indeed NP-hard \cite{martello2000}, so heuristic and approximation algorithms are usually used to find good solutions in a reasonable amount of time.\\

\section{State of art algorithms}
*TODO


\section{Contribution}
*TODO
